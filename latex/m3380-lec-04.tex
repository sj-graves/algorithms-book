\documentclass[m3380-lec-main.tex]{subfiles}
\setcounter{chapter}{3}
\begin{document}

\chapter{Functions and object-oriented programming}

\section*{Goals}
\begin{enumerate}[1.~]\setlength{\itemsep}{0pt}
\item Discuss functions defined both by \verb|def ...:| and as \verb|lambda| functions.
\item Introduce classes.
\item Discuss operator overloading.
\end{enumerate}

\section{Functions in programming}
A mathematical function is a well-defined relation between two sets, if you've taken Foundations or other proof-based courses. A function in computer programming is as well, although it may not often seem like it.

In programming, functions are necessary when you need to repeatedly perform the same algorithm, perhaps on a variety of inputs. Input values to a function are called its \emph{arguments}, and we will use the terminology of \emph{passing arguments to a function}. Every Python function also has an output value, although sometimes its output is \verb|None|.

\subsection{\texttt{lambda} expressions}
The simplest type of function in Python is called a \verb|lambda| expression. Let's say we want to define a function $f$ which takes two arguments, $n$ and $x$, and returns $nx$. Mathematically, this is simple: let $f(n,x)=nx$. When we need to define such a simple function for our own use in a Python program, we can define $f$ in almost the same way.

\smallskip{\fn\begin{verbatim}
>>> f = lambda n,x: n*x
>>> f(4,12)
48
>>> f(4,'12')
'12121212'
>>> f([4],'12')
Traceback (most recent call last):
  File "<pyshell#92>", line 1, in <module>
    f([4],'12')
  File "<pyshell#89>", line 1, in <lambda>
    f = lambda n,x: n*x
TypeError: can't multiply sequence by non-int of type 'str'
\end{verbatim}
}\smallskip\noindent
There is a limitation: \verb|lambda| expressions are not block structures, and if there is more than one expression following the \verb|lambda ... :| a syntax error will occur. In fact, trying to enter an indented block by pressing enter after the colon raises a syntax error!

\smallskip{\fn\begin{verbatim}
>>> g = lambda x:
	
SyntaxError: invalid syntax
\end{verbatim}
}\smallskip\noindent
The variable names which occur after the keyword \verb|lambda| and before the colon are the tuple of arguments to the function; the expression after the colon is what is evaluated based upon those arguments, and whatever is evaluated is returned as the function's output.

\subsection{\texttt{def}ining functions}
The long-form way to define a function, and the way that methods of classes will be defined later in the lesson, is using the \verb|def| command. The syntax is no more complicated than anything else we've worked with so far. Save the following as \verb|les4_ex1.py| and run it in IDLE.

\smallskip{\fn\begin{verbatim}
def fib(n):
    '''A mysterious function appears!

    The argument n must be a positive integer.
'''
    if (type(n)!=type(1)) or (type(n)==type(1) and n<0):
        raise TypeError('function fib expects positive integer')
    seq = [1,1]
    while len(seq)<n:
        seq.append(seq[-1]+seq[-2])
    return seq[-1]
\end{verbatim}
}\smallskip\noindent
You'll notice that when you run this through IDLE, nothing appears to happen! However, something important has occurred: try typing \verb|help(fib)| at the prompt.

\smallskip{\fn\begin{verbatim}
>>> help(fib)
Help on function fib in module __main__:

fib(n)
    A mysterious function appears!
    
    The argument n must be a positive integer.
\end{verbatim}
}\smallskip\noindent 
So we see that now there is a function called \verb|fib|, and if we pass a positive integer to it, something should happen.

Let's look at the code you've saved as \verb|les4_ex1.py| once again. The first expression is \verb|def fib(n):|, which tells Python to define a function named \verb|fib| in the current context and let it have one argument, \verb|n|, which can be used inside the function. Moving into the \verb|def| block, the first expression we encounter is a multiline string! This is very good programming practice in Python, and it's called the \emph{documentation string}, or \emph{docstring} for short. It is good programming style\footnote{Yes, there are such things as good programming style and bad programming style. You can consider it two ways. The first way is that writing code in a nice, consistent style makes it easier for you to later relearn what you've done. The second is that I will have to read code that you write, and if it is horribly difficult to decipher because you haven't been nice to me, I will have a harder time being nice to you! The ``manual" of good style for Python is generally accepted to be what is documented in Python Enhancement Proposal 8 (PEP 8). This can be found at \href{https://www.python.org/dev/peps/pep-0008/}{\texttt{https://www.python.org/dev/peps/pep-0008/}}. A very brief overview of that style guide is found at \href{https://docs.python.org/3/tutorial/controlflow.html\#intermezzo-coding-style}{\texttt{https://docs.python.org/3/tutorial/controlflow.html\#intermezzo-coding-style}}.} to include a docstring, since it enables the \verb|help| function to tell something about the defined function. I won't go into the formatting of the docstring here: see the Python Tutorial.

Moving into the actual code of our function, we see that the very first thing it does is checks the type and the value of the argument, and if the type is wrong or the integer is negative, a \verb|TypeError| is raised. You can think of this as ``bulletproofing" our function: if you're writing code for other people to use, you must be very certain that they cannot through malice or ignorance pass an argument to your function which will fundamentally break stuff\footnote{This is a technical term.}. Type and value checking the input to a function is a good way to do this. If you are writing code just for yourself and know that nothing seriously bad will happen if you accidentally pass a float or a string instead of an integer, you will often skip this step.

The next line instantiates a list, \verb|seq|, and the following \verb|while| loop makes use of the \verb|append| method of the \verb|list| class to lengthen the list to the length passed as argument \verb|n|. Once \verb|len(seq) >= n|, the loop exits and the last entry of \verb|seq| is \verb|return|ed.

Generally speaking, this structure is what every function will follow! You'll pass in some arguments, maybe do some type checking, and then run an algorithm on the arguments. When you complete the algorithm, you return something. One special note is that even functions which contain no \verb|return| statement return a value -- it is the special value \verb|None| of data type \verb|NoneType|. If you try to execute the expression \verb|None|, nothing happens.

\smallskip{\fn\begin{verbatim}
>>> None
>>>
\end{verbatim}
}\smallskip\noindent
You've already used a function which returns \verb|None|: the \verb|print| function.

\smallskip{\fn\begin{verbatim}
>>> print(print(5))
5
None
>>>
\end{verbatim}
}\smallskip\noindent

\section{Classes and object-oriented programming}

I'm not going to go into a deep philosophical discussion about the different ``programming paradigms," of which \emph{object-oriented programming} is one. We'll treat this in a very introductory manner: complicated data types are best considered as a \emph{class} of variables. All the \emph{objects} of the same class should have the same operations which can be performed upon them; these are \emph{methods} of the class. The parameters of objects in the class we will call \emph{attributes}, in keeping with Python terminology.

So we could define a class of complex numbers of the form $a+i b$ where $a,b\in \mathbb{R}$. Arithmetic of complex numbers is different from real numbers, because we have to remember to treat the special number $i=\sqrt{-1}$ accordingly. Mathematically speaking, this means that a complex number $a+ib$ is comprised of an ordered pair of real numbers $(a,b)$, and we must define carefully what are the algorithms for addition, multiplication, etc.

In Python, it makes sense to define a class of complex numbers. Let's start a new file, \verb|les4_ex2.py|. Since we want to be very careful about what a complex number is, we'll start simply.

\smallskip{\fn\begin{verbatim}
class ComplexNumber:
    """A complex number class.
    """
\end{verbatim}
}\smallskip\noindent
If this is all we put in our file, nothing interesting happens except that we can now create a variable of type \verb|ComplexNumber| like so:

\smallskip{\fn\begin{verbatim}
>>> x = ComplexNumber()
>>> type(x)
<class '__main__.ComplexNumber'>
>>> x
<__main__.ComplexNumber object at 0x10367bbe0>
\end{verbatim}
}\smallskip\noindent
This isn't very interesting: when we try to use our \verb|ComplexNumber|, all that Python knows is its type and where in the system's memory it lives. This is because we haven't defined any way for a \verb|ComplexNumber| to be \emph{initialized}. There are two methods that you should always define for a new data class, and initialization is the first of these. In order to know that $x=a+ib$ is a particular complex number, we must know specifically two attributes: its real part $a$ and its imaginary part $b$. Here's how we will do this:

\smallskip{\fn\begin{verbatim}
class ComplexNumber:
    """A complex number class.
    """
    def __init__(self,a,b):
        if not (type(a) in [type(0), type(0.1)] and
                type(b) in [type(0), type(0.1)]):
            raise TypeError("ComplexNumber(a,b) expects a and b to both be of"\
                            + " type 'int' or type 'float'.")
        else:
            self.real_part = a
            self.imag_part = b
\end{verbatim}
}\smallskip\noindent

Here we have a very strange name for a function: \verb|__init__| with three arguments, \verb|self|, \verb|a|, and \verb|b|. The first of these, \verb|self|, is special and should be the first argument of every method of a class. This makes much more sense later when we want to define other methods. We also see that, after the error-checking is done, we assign attributes \verb|real_part| and \verb|imag_part| to \verb|self| by using \verb|self.real_part| and \verb|self.imag_part| as variables. After \verb|__init__| executes, every attribute of the object being initialized must have a value, even if that value is \verb|None|.

Now we need to get to the other mandatory\footnote{Not Python-mandatory, because Python doesn't require this. Instead, it's just extremely bad programming etiquette.} method. This is the \emph{representation} of the object, implemented in Python as the \verb|__repr__| method\footnote{I'm going to pronounce this as ``repper," which rhymes with ``pepper."}. This method must return an unambiguous string representation of the complex number.

For complex numbers, this is incredibly easy. Complex numbers are in bijective correspondence\footnote{Take Foundations!} with ordered pairs of real numbers, so the \emph{canonical} representation of a complex number $x=a+iy$ is as the ordered pair $(a,b)$. Let's extend \verb|les4_ex2.py| to the following:

\smallskip{\fn\begin{verbatim}
class ComplexNumber:
    ...

    def __repr__(self):
        return str( (self.real_part, self.imag_part) )
\end{verbatim}
}\smallskip\noindent
Now if we save and rerun our module, we can actually assign values to a variable of type \verb|ComplexNumber| and also see what is the value of the variable!

\smallskip{\fn\begin{verbatim}
>>> x = ComplexNumber(1,3.4)
>>> x
(1, 3.4)
>>> type(x)
<class '__main__.ComplexNumber'>
\end{verbatim}
}\smallskip\noindent

Optionally, you can define the \verb|__str__| method for the class, which defines a ``human-readable" format for objects in the class. We might do so as follows.

\smallskip{\fn\begin{verbatim}
class ComplexNumber:
    ...
    
    def __str__(self):
        (a, b) = (self.real_part, self.imag_part)
        if a != 0:
            s = str(a)
            if b<0:
                s += ' - I*(' + str(-b) + ')'
            elif b>0:
                s += ' + I*(' + str(b) + ')'
        else:
            if b<0:
                s = '-I*(' + str(b) + ')'
            else:
                s = 'I*(' + str(b) + ')'
        return s
\end{verbatim}
}\smallskip\noindent

Now we can call this function 

\subsection{Adding more methods}
It's very easy to add new methods to a class, since they're just functions defined inside the block structure of the class. Let's say we want to be able to determine the magnitude (called a \emph{norm} in mathematics) of a complex number \verb|x| by calling the method \verb|x.norm()|. We'll need to add one line to the beginning of our module, and then we can add our function \verb|norm|. Here's what your file \verb|les4_ex2.py| should now look like:

\smallskip{\fn\begin{verbatim}
from math import sqrt

class ComplexNumber:
    """A complex number class.
    """
    def __init__(self,a,b):
        if not (type(a) in [type(0), type(0.1)] and
                type(b) in [type(0), type(0.1)]):
            raise TypeError("ComplexNumber(a,b) expects a and b to both be of"\
                            + " type 'int' or type 'float'.")
        else:
            self.real_part = a
            self.imag_part = b

    def __repr__(self):
        return str( (self.real_part, self.imag_part) )

    def __str__(self):
        (a, b) = (self.real_part, self.imag_part)
        if a != 0:
            s = str(a)
            if b<0:
                s += ' - I*(' + str(-b) + ')'
            elif b>0:
                s += ' + I*(' + str(b) + ')'
        else:
            if b<0:
                s = '-I*(' + str(b) + ')'
            else:
                s = 'I*(' + str(b) + ')'
        return s
                
    def norm(self):
        return sqrt(self.real_part**2 + self.imag_part**2)
\end{verbatim}
}\smallskip\noindent

The line at the beginning, \verb|from math import sqrt|, tells Python to find a module named \verb|math| in either the current directory or Python's module search, and allow us to access the function \verb|sqrt|. Note that if we had already defined a function or variable as \verb|sqrt|, this import command would have overwritten our function or variable! To work around this, we could instead have had our first line be simply \verb|import math|, but then when we wanted to use the function \verb|sqrt| from the \verb|math| module, we would have needed to type \verb|return math.sqrt(self.real_part**2 + self.imag_part**2)|.

What else should we define for complex numbers? It would be nice to be able to access the real and imaginary parts of a complex number, so we'll define \verb|re| and \verb|im| methods. 

\smallskip{\fn\begin{verbatim}
from math import sqrt

class ComplexNumber:
    ...
    def re(self):
        return self.real_part

    def im(self):
        return self.imag_part
\end{verbatim}
}\smallskip\noindent
We should include our arithmetic operations as well. To keep things simple, we'll only define addition and multiplication here, and leave the others as an exercise. Python understands that \verb|3 + 5| and \verb|5 + 3| should both be 8, but maybe we want to be lazy with a variables \verb|x| and \verb|y| of type \verb|ComplexNumber| and have all of \verb|x + y|, \verb|3 + x|, \verb|x + 3|, \verb|3.1 + x|, and \verb|x + 3.1| execute correctly. Interestingly, \verb|3 + x| and \verb|3.1 + x| will be handled identically, but \verb|x + y|, \verb|3 + x|, and \verb|x + 3| are all distinct!

\section{Power tool: operator overloading}
Operator overloading is easy for a human and hard for a computer. A person can immediately see the symbol \verb|+| and know from the context how to interpret it. For instance, we can immediately understand that there is a difference between ``$3+5$" and ``Alice + Bob." A computer has to be told how to handle each of them separately! This is handled by \emph{overloading} the operator of \verb|+|. Python does this behind the curtain. Every data type for which \verb|+| makes sense must have in its class definition a method called \verb|__add__|. In fact, usually it is implemented as \verb|def __add__(self, other):|, and then the return value is \verb|self + other|. It is generally also smart to define a \verb|__radd__| method to return \verb|other + self|, and for numerical quantities they should be equal\footnote{Addition is almost universally a commutative operation.}. 

\subsection{Complex addition} What is the algorithm for complex addition? That is, if I defined $x=a+ib$ and $y=c+id$ for some $a,b,c,d\in\mathbb{R}$, what does it mean to write $x+y$? If we treat $x=a+ib$ and $y=c+id$ as polynomials in variable $i$ with real coefficients, we can use the rules for polynomials and obtain
\begin{align*}
x + y & = (a+ib) + (c+id) \\
	&= a+c+ib+id \\
	&=(a+c) + i(b+d).
\end{align*}
Since it makes sense for complex numbers and real numbers to follow the same rules, this is a sensible definition for the algorithm of addition.
\begin{alg}(Complex Addition)
Let \verb|x| be a \verb|ComplexNumber| with representation \verb|(a,b)| and let \verb|y| be some other variable.
\begin{enumerate}\setlength{\itemsep}{0pt}
\item If \verb|y| is a \verb|ComplexNumber| with representation \verb|(c,d)|, return \\ \verb|ComplexNumber(a + c, b + d)|.
\item If not, but \verb|y| is either an \verb|int| or a \verb|float|, return \\ \verb|ComplexNumber(a + y, b)|.
\item Otherwise, raise a \verb|TypeError|.
\end{enumerate}
\end{alg}


For complex numbers, we update \verb|les4_ex2.py| as follows to implement addition and right addition.

\smallskip{\fn\begin{verbatim}
from math import sqrt

class ComplexNumber:
    ...
    def __add__(self,other):
        if type(other)==type(self):
            return ComplexNumber(self.real_part + other.re(),
                                 self.imag_part + other.im())
        elif type(other) in [type(1), type(0.1)]:
            return ComplexNumber(other + self.real_part, self.imag_part)

    def __radd__(self, other):
        return self.__add__(other)
\end{verbatim}
}\smallskip\noindent

\subsection{Complex Multiplication}
Complex multiplication, like complex addition, follows naturally from the behavior of polynomials with real coefficients. Again letting $x=a+ib$ and $y=c+id$ be complex numbers, we see that
\begin{align*}
xy &= (a+ib)(c+id) \\
	&= ac +ibc +iad + i^2bd \\
	&= (ac-bd)+i(bc+ad).
\end{align*}
If it so happens that $d=0$, then $xy = ac+ibc$. Again, we can use this to determine the algorithm for complex multiplication.

\begin{alg}(Complex Multiplication) Let \verb|x| be a \verb|ComplexNumber| with representation \verb|(a,b)| and let \verb|y| be some other variable.
\begin{enumerate}\setlength{\itemsep}{0pt}
\item If \verb|y| is a \verb|ComplexNumber|, return \\
	\verb|ComplexNumber(a * c - b * d, a * d + b * c)|
\item If not, but \verb|y| is either an \verb|int| or a \verb|float|, return\\
	\verb|ComplexNumber(a*y, b*y)|
\item Otherwise, raise a \verb|TypeError|.
\end{enumerate}
\end{alg}

To implement this is similar to the implementation of addition: there are builtin operators for both left and right multiplication (which is important since multiplication is often not commutative) which can be overloaded, named \verb|__mul__| and \verb|__rmul__|. Including them, here is the final, full version of \verb|les4_ex2.py|.

\smallskip{\fn\begin{verbatim}
from math import sqrt

class ComplexNumber:
    """A complex number class.
    """
    def __init__(self,a,b):
        if not (type(a) in [type(0), type(0.1)] and
                type(b) in [type(0), type(0.1)]):
            raise TypeError("ComplexNumber(a,b) expects a and b to both be of"\
                            + " type 'int' or type 'float'.")
        else:
            self.real_part = a
            self.imag_part = b

    def __repr__(self):
        return str( (self.real_part, self.imag_part) )

    def __str__(self):
        (a, b) = (self.real_part, self.imag_part)
        if a != 0:
            s = str(a)
            if b<0:
                s += ' - I*(' + str(-b) + ')'
            elif b>0:
                s += ' + I*(' + str(b) + ')'
        else:
            if b<0:
                s = '-I*(' + str(b) + ')'
            else:
                s = 'I*(' + str(b) + ')'
        return s
                
    def norm(self):
        return sqrt(self.real_part**2 + self.imag_part**2)

    def re(self):
        return self.real_part

    def im(self):
        return self.imag_part

    def __add__(self,other):
        (a, b) = (self.real_part, self.img_part)
        if type(other)==type(self):
            (c, d) = (other.re(), other.im())
            return ComplexNumber(a + c, b + d)
        elif type(other) in [type(1), type(0.1)]:
            return ComplexNumber(other + a, b)
        else:
            raise TypeError("No method is defined for addition of "\
                            + "ComplexNumber + '" \
                            + other.__class__().__name__() + "'")

    def __radd__(self, other):
        return self.__add__(other)

    def __mul__(self, other):
        (a, b) = (self.real_part, self.imag_part)
        if type(other)==type(self):
            (c, d) = (other.re(), other.im())
            return ComplexNumber(a * c - b * d, a * d + b * c)
        elif type(other) in [type(0), type(0.1)]:
            return ComplexNumber(a * other, b * other)
        else:
            raise TypeError("No method is defined for addition of "\
                            + "ComplexNumber + '" \
                            + other.__class__().__name__() + "'")

    def __rmul__(self, other):
        return self.__mul__(other)
\end{verbatim}
}\smallskip\noindent


\subsection{Overloading other operators} There are many, many operators and methods which you will want to overload to make robust classes. In fact, almost everything can be overloaded if you can determine how Python handles an operator behind the scenes. For numerical data types, there's an easy way to see what the ``normal" operators are: type \verb|help(float)| and read the documentation for the \verb|float| class! In fact, you can often inherit methods from ``parent" classes if the operations are already correct.

\subsection{``It's a trap!"} In fact, we didn't need to do any of this work to implement complex numbers. Try \verb|help(complex)|. Following a convention from engineering and the sciences, $j=\sqrt{-1}$, and entering \verb|z = 5-3j| produces a complex number with full functionality. We won't be saving our \verb|ComplexNumber| class for further use.



\end{document}